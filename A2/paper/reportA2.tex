\documentclass{article} % For LaTeX2e
\usepackage{nips14submit_e,times}
\usepackage{hyperref}
\usepackage{url}
%\documentstyle[nips14submit_09,times,art10]{article} % For LaTeX 2.09


\title{A2 homework submission \\ Team: the gurecki \\ Deep Learning 2015, Spring}


\author{
David Halpern\\
Department of Psychology\\
New York University\\
\texttt{david.halpern@nyu.edu} \\
\And
Anselm Rothe\\
Department of Psychology\\
New York University\\
\texttt{ar3918@nyu.edu} \\
\AND
Alex Rich\\
Department of Psychology\\
New York University\\
\texttt{asr443@nyu.edu} \\
}

\newcommand{\fix}{\marginpar{FIX}}
\newcommand{\new}{\marginpar{NEW}}

\nipsfinalcopy % camera-ready version

\begin{document}


\maketitle

\section{Input}
\section{Architecture}
\subsection{Stage 1}
\subsection{Stage 2}
\subsection{Stage 3}
\section{Learning Techniques}
\section{Learning Procedure}

\section{Architecture}

\subsection{Input}

The input is a 3D array with three 2D feature maps of size $32 \times 32$. We preprocessed the training data 
by normalizing across features so that each feature had a mean of 0 and a standard deviation of 1 and then
used the same normalization for the test data. The data was also normed per pixel so each pixel across images
had a mean of 0 and a standard deviation of 1. Each image is padded with zeros of size $2\times2\times2\times2$
in order to make for better convolutional filter learning.

\subsection{Model}
The first layer applies 23 $7\times7$ convolutional filters with a stride of 2.  These 
filters have a ReLU activation function and are fed into a max pooling layer which 
takes the max of each $3\times3$ region with a step size of $2\times2$. Each filter
is now $22\times22$. This they are all fed into a linear layer with 50 outputs nodes.
The outputs are then run through a Tanh activation function and finally put through 
another linear layer with 10 output nodes, one for each category. The model then
uses log softmax to convert energies into choices.

\section{Learning Techniques}

We used dropout to train the convolutional layers.

\section{Training Procedure}

We used the to train our model. The model was trained 
using stochastic gradient descent with a learning rate of .001 ???,
a batch size of 128 a momentum of .9 and weight decay of ???.  We
used the typical classNLLcriterion as the loss function. We used a validation set of 500 
of the original 5000 labeled training examples. The optimization procedure was run for 
100 epochs over the remaining 4500 training data.

Performance ???

\section{Experiments}

These experiments did not help to boost the performance and thus where not included in our final submission.

\subsection{Unlabeled data}
We attempted to use uniform prescription to train our model using the unlabeled data. In order to 
do this, we used KL Distance as our loss function so that the target for the unlabeled data could be
a uniform distribution on all of the categories. The targets for the labeled data were a delta function
with all probability mass on the true category. We tried using various amounts of unlabeled data from
500 to all 100000 but this did not seem to improve performance

\subsection{Data augmentation}
Following Dosovitskiy et al. (2014), we converted all images into HSV color space. We inflated the train data set by a factor of 3 and applied each of the following transformations (thereby flipping a 0.5 coin to decide whether or not to apply the transformation):
\begin{enumerate}
	\item A up-left or right-down shift by 2-10 pixels (randomly determined)
	\item A horizontal flip
	\item A rotation by 6-10 degrees clockwise or counterclockwise
\end{enumerate}

\subsubsection*{References}

\small{
[1] Dosovitskiy, A., Springenberg, J. T., Riedmiller, M., \& Brox, T. (2014). Discriminative unsupervised feature learning with convolutional neural networks. {\it Advances in Neural Information Processing Systems}, pp. 766-774.

[2] Xiang Zhang (2015). {\it How to Train STL-10 Well}. http://goo.gl/xJGvyH
}


\end{document}
